\documentclass[11pt]{article}
\usepackage{geometry}                % See geometry.pdf to learn the layout options. There are lots.
\geometry{letterpaper}                   % ... or a4paper or a5paper or ... 
%\geometry{landscape}                % Activate for for rotated page geometry
%\usepackage[parfill]{parskip}    % Activate to begin paragraphs with an empty line rather than an indent
\usepackage{graphicx}
\usepackage{amssymb}
\usepackage{epstopdf}
\usepackage{amsmath}
\usepackage[version=4]{mhchem}
\usepackage{float}
\DeclareGraphicsRule{.tif}{png}{.png}{`convert #1 `dirname #1`/`basename #1 .tif`.png}
\usepackage[utf8]{inputenc}
\usepackage{fontspec}
\usepackage{pgfplots}
 \usepackage[usenames,dvipsnames]{pstricks}
\usepackage{epsfig}
 \usepackage{pst-grad} % For gradients
 \usepackage{pst-plot} % For axes
 \usepackage{pst-eucl}

 \usepackage[space]{grffile} % For spaces in paths
 \usepackage{etoolbox} % For spaces in paths
\makeatletter % For spaces in paths
 \patchcmd\Gread@eps{\@inputcheck#1 }{\@inputcheck"#1"\relax}{}{}
\makeatother
\usepackage{pst-node,pst-circ}

\setlength{\parskip}{\baselineskip}%
\setlength{\parindent}{0pt}%

\title{BMAT 2010 Solutions}
\author{}
\date{}                                           % Activate to display a given date or no date

\begin{document}
\maketitle
\section*{Q3}
Over 12 minutes the sample slowly decays to 16 mg of uranium-234.  This implies there were 16 mg of protactinium-234 before the $\beta$-emission.  Looking at the table, after 1.2 minutes there is 8 mg of uranium implying there is 8 mg of protactinium; its mass has halved.  Again, looking at 2.4 minutes there is 12 mg of uranium and therefore 4 mg of protactinium.  It takes 1.2 minutes for the mass of the protactinium to half and therefore, this is its half-life, Option \textbf{A}.

\section*{Q4}
Immediately we need to think about what we've been given in the question.  The values p and q represent the fraction of water filled for the larger and smaller container respectively.  So initially p=1 and q=0.  Now we pour half of the larger container into the smaller so p, by definition, will be equal to 0.5.  This also indicates that q > 0.5 as its smaller than container p but contains the same amount of water.  Our answer is therefore Option \textbf{C}.

\section*{Q6}
Lets just follow the mass through each equation.  So, 12g of carbon is used in stage 2, which results in 56 g of carbon monoxide being produced This is calculated by adding the masses $2\times(12 + 16)=56$.  In stage 3, for every molecule of carbon-monoxide reacted, a molecule of carbon-dioxide is produced.   

Therefore the amount of carbon-dioxide is calculated by adding another oxygen particle to the CO produced in stage 2, so $2 \times (12+16+16) = 88$.  Therefore our answer is Option \textbf{C}.

\section*{Q7}
This is testing your ability to read the values of frequency and amplitude from a graph.  Remember that the definition of amplitude is the height of the wave from the average position to a peak and the frequency is the number of waves per second.

So, lets calculate the amplitude.  The average of this wave, halfway between two peaks, is 13 metres.  We have a maximal point at 16 m, so 16-13 = 3 m for the amplitude.

The time taken for a full wave to pass is from 0 hours to 12 hours, so our wavelength is 12 h or $12 \times 3600$ s.  The frequency is the number of full wavelengths per second, so just take the reciprocal, which results in $1/(12\times3600)$.  

We have our results, which correspond to Option \textbf{A}.

\section*{Q8}
This question is just asking what the probability of getting a particular results is.  To see this imagine we make a first move from position A to position B.  We want to return to position A on the first move, which is only allowed by the opposite direction and same distance, so no matter what our original move is, its possible to reverse it.

Since we have a $\frac{1}{4}$ probability of getting a particular distance, and the same for a particular direction, the probability is just the product between them
\begin{equation*}
\frac{1}{4} \times \frac{1}{4} = \frac{1}{16},
\end{equation*}
which is Option \textbf{C}.

\section*{Q11}
Remember what happens with $\beta$ and $\alpha$ emission.  

In $\alpha$-decay a He nucleus is emitted resulting in a loss of 2 to the atomic number and 4 to the mass number.  Since we've had 3 of these emitted we need to subtract 12 from the mass number and 6 from the atomic number.  So we currently sit at 
\begin{equation*}
\mathrm{^{207}_{80}X}.
\end{equation*}

In $\beta$-decay a neutron transforms to a proton, emitting an electron.  This means no change to the mass number as neutrons and protons are the same mass, but results in 1 being added to the atomic number.  We have two of these decays so we add 2 to the atomic number, resulting in,
\begin{equation*}
\mathrm{^{207}_{82}X}.
\end{equation*}

Our solution is therefore Option \textbf{C}.

\section*{Q12}
\section*{Q15}
When switch P is open and switch Q is closed, almost no current will flow through any bulbs except bulb Y.  This is because the majority of the current will short circuit through switch Q.  

Now we open switch Q and close switch P.  A large amount of current can now short circuit through switch P, so we can view it as a series circuit with just bulbs X and Y.  Since bulb X initially had very little current flowing through it, it will be brighter, and since bulb Y is now in a series circuit and is not the only bulb, it will be dimmer than initially.  

The solution is therefore Option \textbf{B}.

\section*{Q17}

Since X carries the condition, their parents U and V must both be carriers.  Therefore U is 100\% a carrier.

If P and Q are both carriers of a recessive allele, the probability of the rest of their children carrying the allele is 50\%, as none of their children carry the condition, so will only have one copy of the recessive allele.  The only combinations of alleles allowed are RD and DD, for a dominant allele D and recessive R.  This means S and T have a 50\% chance of being carriers.

The solution is Option \textbf{E}.

\section*{Q19}
Take each graph in turn.
\subsection*{P}
This is a velocity-time graph, so to calculate the acceleration, we calculate the gradient.  Remember, $dy/dx$ for a straight line, which is $10/24 \mathrm{m/s^2}$, which does not equal 2.4. 

\subsection*{Q}
It's difficult to see the exact point where the guiding points meet the y-axis but approximately 58 $m/s$ is not a bad guess.  Taking $dy/dx$ again, as it's a velocity-time graph, results in 
\begin{equation*}
\frac{58-10}{20}=2.4.
\end{equation*}
Therefore Q is accelerating at $2.4 \mathrm{m/s^{2}}$.

\subsection*{R}
This is a distance time graph, but it has a constant gradient.  So taking a differential twice would just equal 0.  So this isn't accelerating.

\subsection*{S}
This is the same as R.

Since only Q is accelerating at 2.4 $\mathrm{m/s^2}$, the solution is Option \textbf{B}.

\section*{Q20}
Equate the equations for the surface area and the volume and rearrange to get in terms of h.

\begin{equation*}
\begin{aligned}
\pi r^2 h &= \pi (2rh + 2r^2)\\
r^2h &= 2rh + 2r^2\\
h(r-2) &= 2r\\
h &= \frac{2r}{r-2}\\
\end{aligned}
\end{equation*}

This gives us our solution, which is Option \textbf{A}.


\section*{Q23}
For this we use the equations for gravitational potential energy, $E_p = mgh$, and kinetic energy, $E_k = \frac{1}{2}mv^2$.

5kg of water passing through each second will reach a height of 5m.  This means that our gravitational potential energy is $5\times5\times10 = 250$ J.  Immediately, we know that this will be the power of the pump, as every second the water is being given 250 J of energy.  

Upon leaving the pump, the water will have 250 J of kinetic energy, and rearranging the equation for kinetic energy
\begin{equation*}
v = \sqrt{\frac{2E_k}{m}} = \sqrt{\frac{500}{5}} = 10,
\end{equation*}
so it leaves at 10 m/s.  Our solution is therefore Option \textbf{G}.

\section*{Q24}
For this we'll need to calculate the area of the second square as a fraction of the first.  

To do this, look at the parts of the original square not covered up by the second square.  We effectively have 4 triangles in each corner of length 1/3 and 2/3 in relation to the original square, with their hypotenuse being the length of a side of the second square.  So, using pythagoras' theorem, $a^2+b^2=c^2$, we can calculate this length,
\begin{equation*}
c^2 = \frac{1}{3}^2 + \frac{2}{3}^2 = \frac{5}{9}.
\end{equation*} 
This is conveniently the area of the second square as a fraction of the original.  If we want to calculate the area of the fourth square following this process, we just apply 5/9 two more times.

\begin{equation*}
1 \times \frac{5}{9} \times \frac{5}{9} \times \frac{5}{9} = \frac{125}{729}
\end{equation*}

Therefore our answer is Option \textbf{C}.

\section*{Q27}
For this you'll need the equation for work done $E=F\times d$.

For this, the car is working against both friction and gravity, so the work done to oppose friction will be $50\times500$, which is 25 kJ.  

It will have also done work to oppose gravity, which we can work out using gravitational potential energy gained, $E_p=mgh$.  This is $800\times 10\times 2.5 = 20 \mathrm{kJ}$ as it gains 1 metre up for every 20 metres across.

Summing these up gives us the total work done by the engine of 45 kJ, so the solution is Option \textbf{D}.



\end{document}  